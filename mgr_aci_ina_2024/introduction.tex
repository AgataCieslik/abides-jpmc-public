\chapter*{Wstęp}

Głównym celem pracy jest zbudowanie modelu rynku z uwzględnieniem zróżnicowania wiedzy na temat handlowanego instrumentu w oparciu o istniejący model referencyjny. Do problemu podejdziemy metodycznie, w pierwszej kolejności analizując sposób działania i specyfikę rynku, nakreślając przy tym problemy, jakie rzeczywiści gracze adresują w konstrukcji swoich strategii inwestycyjnych oraz podstawowe stylizowane fakty na temat rynku.

Mając dany sformalizowany opis rynku przejdziemy do przeglądu dotychczasowego dorobku w dziedzinie konstrukcji agentów modeli ekonomicznych, ze szczególnym wyróżnieniem konceptów wykorzystanych później do rozwijania nowych typów agentów. 

Ostatnim etapem przed opracowaniem finalnego modelu będzie zapoznanie się z modelem referencyjnym - odwzorowaniem kluczowych konceptów agentowych modeli finansowych w wybranej implementacji. Wykorzystując punkt odniesienia w postaci modelu referencyjnego przejdziemy do kluczowego  elementu pracy - konstrukcji modelu z uwzględnieniem przewagi informacyjnej części graczy oraz zastosowanie go do weryfikacji hipotez na temat optymalnego wykorzystania przewagi informacyjnej.
