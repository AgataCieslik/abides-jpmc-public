\chapter*{Podsumowanie}

Podejście do rozwinięcia modelu agentowego rynku okazało się, nieoczekiwanie, zadaniem wymagającym i interdyscyplinarnym. Powierzchownie nieskomplikowane konstrukcje agentów zawierają elementy zaadaptowane z różnych, często nie wprost powiązanych, dziedzin. 

Po zapoznaniu się z dotychczasowym dorobkiem w dziedzinie modelowania agentowego rynku zdecydowaliśmy się rozwinąć wybrany model referencyjny w kierunku odtworzenia obserwowanego współcześnie zróżnicowanego dostępu do informacji. Zaproponowaliśmy model zbudowany wokół popartego stylizowanymi faktami założenia o bezpośrednim wpływie zdarzeń ekonomicznych na cenę, wprowadzając przy tym dwa nowe podtypy agentów oraz system komunikacji między nimi. Rozwinięty model wykorzystaliśmy do weryfikacji hipotez na temat skutków działań gracza z uprzywilejowaną wiedzą. 

Opisany w pracy model skoncentrowany wokół zdarzenia ekonomicznego oraz towarzyszące mu symulacje i ich analiza w żadnym razie nie wyczerpują tematu nierówności dostępu do informacji na rynkach finansowych. Wręcz przeciwnie - wyniki przeprowadzonych symulacji sygnalizują nowe wątki domagające się rozwinięcia i szerszej analizy. 